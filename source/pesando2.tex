\subsection{Proiezione stereografica}
Per un cerchio, valgono le relazioni seguenti:
\[ x_{N,S} = R \frac{\sin\theta}{1\mp\cos\theta} \]
Nel caso di una sfera, in coordinate polari tale relazione varra' per il modulo rispetto alla latitudine; la longitudine dara' invece l'argomento. Riscrivendo in campo complesso:
\[ z_{N,S} = \rho_{N,S} e^{i\varphi} \]
con 
\[ \begin{cases}
	\rho_{N,S} = R \frac{\sin\theta}{1\mp\cos\theta} & \\
	\varphi_{N,S} = \phi & \\
   \end{cases}
\]


Usando queste relazioni, si trova facilmente che 
\[ \rho_N = \frac{1}{\rho_S} \]
Tale relazione \`e olomorfa in \( U_N \cap U_S \), cio\`e il piano complesso senza l'origine e l'infinito.
Le coordinate $w_{N,S}$, ottenute proiettando sul piano tangente alla sfera nel polo opposto, hanno modulo doppio delle rispettive $z_{N,S}$:
\[ w_{N,S} = 2z_{N,S} \]

\subsection{Rotazioni}
I calcoli di questo paragrafo si intendono fatti con $z_N$. Le derivate si considerano per ora applicate a funzioni olomorfe.
\[ L_z = -i\dpartial{}{\phi} = -i \dpartial{\varphi}{\phi}\dpartial{z}{\varphi}\dpartial{}{z} = z\dpartial{}{z} \]
\[ L_\pm = \pm e^{\pm i\phi} (\dpartial{}{\theta} \pm i \cot\theta \dpartial{}{\phi} ) \]
\[ = \pm e^{\pm i\varphi}Re^{i\varphi}  (-\frac{1}{1- \cos\theta} \mp  \cot\theta \frac{\sin\theta}{1-\cos\theta})\dpartial{}{z}  \]
Siccome \( \rho_N = \frac{1+c}{1-c}\),
\[ L_+ = -z^2\dpartial{}{z} , \;\;\;\;\;\;  L_- = \dpartial{}{z} \]
\[ [L_+,L_-] = -z^2\dpartial{^2}{z^2} + 2z\dpartial{}{z} + z^2\dpartial{^2}{z^2} = 2L_z \]

\subsection{Autofunzioni di $L_z$}
\subsubsection*{Funzioni non olomorfe}
Cercando come autofunzione una funzione \( F_m(z_N,\bar{z_N}) = z_N^m f(z_N\bar{z_N}) \), che dipende anche dalla variabile complessa coniugata e non e' quindi olomorfa, e' necessario pensare le derivate su z come derivate sulle coordinate $(\theta,\phi)$, di cui poi si esegua un push-forward [\todo e' giusto?] sulle coordinate z. In formule,
\[ \dpartial{}{\theta} \rightarrow \left.\dpartial{}{z}\right|^R ,\;\;\;\;\;\; \dpartial{}{\phi} \rightarrow \left.\dpartial{}{z}\right|^F \]
dove
\[ \left.\dpartial{g(z)}{z}\right|^{R,F} = lim_{|h|\rightarrow 0} \frac{g(z+|h|e^{i\phi_{R,F}}) - g(z)}{|h|e^{i\phi_{R,F}}} \]
Con \( \phi_R = \varphi \) e \( \phi_F = \varphi+\pi/2 \).
Quindi per funzioni non olomorfe
\[ L_z = z\left.\dpartial{}{z}\right|^F \]
e usando la relazione
\[ \cos\theta = \frac{\rho^2 - 1}{\rho^2 +1} \rightarrow \frac{1}{1-\cos\theta} = \rho^2 +1\]
si trova
\begin{equation} \label{eq:L+-}
 L_\pm = \mp \frac{z^2}{|z|^2}\frac{R}{2} \left[ (|z|^2+1)\left.\dpartial{}{z}\right|^R \pm (|z|^2-1)\left.\dpartial{}{z}\right|^F     \right]
\end{equation}

\subsubsection*{Autovalori di $F_m$}
Fatte queste premesse,
\[ L_z F_m = \]
\[mz_N^{m-1}f(\rho^2) + z_N^m \dpartial{f}{(\rho^2)} lim_{|h|\rightarrow 0} \frac{(z+|h|e^{i(\varphi+\pi/2)})(\bar{z}+|h|e^{-i(\varphi+\pi/2)})-z\bar{z}}{|h|e^{i(\varphi+\pi/2)}} \]
\[= mz_N^{m-1}f(\rho^2) + \dpartial{f}{(\rho^2)}(-\bar{z}+\bar{z}) = mz_N^{m-1}f(\rho^2)\]

\subsubsection*{Forma di $Y_l^l$}
Poich\'e, come visto, \(\left.\dpartial{}{z}\right|^F f(\rho^2) = 0\),
\[ L_+F_m = -z^2\dpartial{z^m}{z}f(\rho^2) - z^m \frac{z^2}{|z|^2} (|z|^2+1) \frac{R}{2} \left.\dpartial{z\bar{z}}{z}\right|^R \dpartial{f}{\rho^2} \]
e, dato che \( \left.\dpartial{z\bar{z}}{z}\right|^R = 2\bar{z} \) (per la definizione data),
\[ L_+F_m = -z^{m+1} \left[ mf + R(|z|^2+1) \dpartial{f}{\rho^2}   \right] \]
Imponendo l'annullamento della parentesi si trova un'equazione differenziale che ha soluzione
\[ f_m (\rho^2) = f_0\left(\frac{\rho^2 +1}{\rho_0^2+1}\right)^{-m/R} \]
Definendo 
\[ Y_l^m = z^m f_0\left(\frac{\rho^2 +1}{\rho_0^2+1}\right)^{-m/R}  \]
si vede infine che 
\[ L_+Y^m_l = 0 \iff m=l \]
Da \ref{eq:L+-} si trova subito che 
\[ L_-Y_l^l = -2lzY_l^l\]

