\subsection{Proiezione stereografica}
Per un cerchio, valgono le relazioni seguenti:
\[ x_{N,S} = R \frac{\sin\theta}{1\mp\cos\theta} \]
Nel caso di una sfera, in coordinate polari tale relazione varra' per il modulo rispetto alla latitudine; la longitudine dara' invece l'argomento. Riscrivendo in campo complesso:
\[ z_{N,S} = \rho_{N,S} e^{i\varphi} \]
\[ \bar{z}_{N,S} = \rho_{N,S} e^{-i\varphi} \]
con 
\[ \begin{cases}
	\rho_{N,S} = R \frac{\sin\theta}{1\mp\cos\theta} & \\
	\varphi_{N,S} = \phi & \\
   \end{cases}
\]


Si definisce la funzione che lega le due carte nel modo seguente:
\[ z_N = \frac{1}{z_S} \]
\[ \bar{z}_N = \frac{1}{\bar{z}_S} \]
Tale relazione \`e olomorfa in \( U_N \cap U_S \), cio\`e il piano complesso senza l'origine e l'infinito.
Le coordinate $w_{N,S}$, ottenute proiettando sul piano tangente alla sfera nel polo opposto, hanno modulo doppio delle rispettive $z_{N,S}$:
\[ w_{N,S} = 2z_{N,S} \]

\subsection{Rotazioni}
I calcoli di questo paragrafo si intendono fatti con $z_N$. % Le derivate si considerano per ora applicate a funzioni olomorfe.
\[ L_z = -i\dpartial{}{\phi} = -i \dpartial{\varphi}{\phi}(\dpartial{z}{\varphi}\dpartial{}{z} + \dpartial{\bar{z}}{\varphi}\dpartial{}{\bar{z}}) = z\dpartial{}{z} - \bar{z}\dpartial{}{\bar{z}}\]
\[ L_\pm = \pm e^{\pm i\phi} (\dpartial{}{\theta} \pm i \cot\theta \dpartial{}{\phi} ) \]
\[ = \pm e^{\pm i\varphi}Re^{i\varphi}  (-\frac{1}{1- \cos\theta} \mp  \cot\theta \frac{\sin\theta}{1-\cos\theta})\dpartial{}{z} \pm e^{\pm i\varphi}Re^{-i\varphi}  (-\frac{1}{1- \cos\theta} \pm  \cot\theta \frac{\sin\theta}{1-\cos\theta})\dpartial{}{\bar{z}} \]

Siccome \( z\bar{z} = \frac{1+c}{1-c}\),
\[ L_+ = -\frac{z^2}{R}\dpartial{}{z} -R\dpartial{}{\bar{z}} , \;\;\;\;\;\;  L_- = R\dpartial{}{z} + \frac{\bar{z}^2}{R}\dpartial{}{\bar{z}} \]
Usando che \( \frac{\partial^2}{\partial z \partial \bar{z}} = \frac{\partial^2}{\partial \bar{z} \partial z}\) 
\[ [L_+,L_-] = 2L_z \]



\subsubsection*{Autovalori di $F_m$}
Tenendo conto che 
\[ L_z f(\rho) = (z\dpartial{\rho}{z} - \bar{z}\dpartial{\rho}{\bar{z}}) \dpartial{f}{\rho} = 0\]
\`e immediato verificare che 
\[ L_z F_m = L_z[z^m] f(\rho) + z^m L_z[f(\rho)] = m F_m\]


\subsubsection*{Forma di $Y_l^l$}
\[ L_+ [f(z\bar{z})] = (-z^2\dpartial{\rho}{z} - \dpartial{\rho}{\bar{z}} ) \dpartial{f}{\rho} \]
\[ = -\frac{z^{m+1}}{R} \left[ mf + (|z|^2+R^2) \dpartial{f}{\rho^2}   \right] \]
Imponendo l'annullamento della parentesi si trova un'equazione differenziale che ha soluzione
\[ f_m (\rho^2) = f_m(\rho^2 + R^2)^{-m} \]
Definendo 
\[ Y_l^m = z^m f_m(\rho^2 + R^2)^{-m}  \]
si vede infine che 
\[ L_+Y^m_l = 0 \iff m=l \]
\[ L_-Y_l^l = (\frac{mR}{z} +\frac{\bar{z}}{R}) Y_l^l \]

Nelle coordinate $(z_S, \bar{z}_S)$
\[ Y_l^m = z_S^{-m}f_m((z_S\bar{z}_S)^{-1} + R^2)^{-m}  \]


