\subsection{ Osservatori uniformemente accelerati}
Si considera una particella $P_0$, uniformemente accelerata rispetto al sistema istantaneamente inerziale  $\mathcal{I}$.
Il sistema $\mathcal{I}$ va inteso come un insieme di sistemi di riferimento inerziali, tra i quali, per ogni tempo, si considera quello rispetto a cui la particella $P_0$ \`e istantaneamente ferma.
\subsubsection{Particella accelerata}
Si considera una generica particella $P$, con velocit\`a $u$ e accelerazione $a$ nel sistema $\mathcal{I}$.
Il boost a velocit\`a inversa dal sistema in movimento $\mathcal{I}$ al sistema terra $\mathcal{T}$ \`e
\begin{equation}
	\begin{cases}
		\dx_T = \gamma (\dx + v \dt) &  \\
		\dt_T = \gamma (\dt + v \dt) &  \\ 
	\end{cases}
\end{equation}
\[  u_T = \frac{\dx_T}{\dt_T} = \frac{u+v}{1+uv} \]
\[ \de u_T = \frac{1-v^2}{(1+uv)^2} \de u \]
\begin{equation} \label{accel_P}
	a_T = \frac{\de u_T}{\dt} = \frac{(1-v^2)^{3/2}}{(1+uv)^3} a 
\end{equation}
Volendo trovare la velocit\`a della particella $P_0$, si utilizza l'equazione \ref{accel_P} e la si specializza: la velocit\`a $u$ va posta nulla perch\`e si considera la particella $P_0$, ferma in $\mathcal{I}$, e l'evoluzione temporale di $v(t_T)$, rispetto al sistema $\mathcal{T}$; la velocit\`a della particella rispetto a $\mathcal{T}$ \`e adesso $v$ e \( a = a_0 \):
\[ a_T = \frac{\de v(t_T)}{\dt_T} = [1-v^2(t_T)]^{3/2} a_0 \]
\[ a_0t_T = \int_0^{v(t_T)} \frac{\de v}{(1-v^2)^{3/2}} \]
sostituisco $v=\sin(\theta)$
\[ a_0t_T= \int_0^{\arcsin(v(t_T))} \frac{\de \theta}{\cos^2(\theta)} = \int \de\tan\theta = \tan(\arcsin(v(t_T))) \]
\[ v(t_T) = \sin(\arctan(a_0t_T)) \]
\begin{equation} \label{veloc}
		v(t_T) = \frac{a_0 t_T}{\sqrt{1+(a_0t_T)^2}} 
\end{equation}

Per trovare la legge oraria, considerando che \(x_T(0)=v_T(0)=0\),
\[ \int_{x_T(0)}^{x_T(t_T)} \dx_T = \int_0^{t_T} \frac{a_0t_T}{\sqrt{a+(a_0t_T)^2}} \dt_T \]
Sostituendo prima \( y=a_0t_T \) e poi \( y = \sinh(z) \) si ottiene
\[ x_T(t_T) = \frac{1}{a_0} \int_{y_0}^y \frac{y}{\sqrt{1+y^2}} \de y = \frac{1}{a_0} \int_{z_0}^z \sinh(z) \de z \]
\[ = \frac{1}{a_0} [\cosh\sinh^{-1}(y) - \cosh\sinh^{-1}(y_0)]  = \frac{1}{a_0} [\cosh(\ln(y + \sqrt{1+y^2}) -1]\]
\[ = \frac{y^2+y\sqrt{1+y^2}+1-y-\sqrt{1+y^2}}{a_0(y+\sqrt{1+y^2})}  = \frac{\sqrt{1+y^2}-1}{a_0} \]
\begin{equation} \label{leggeoraria}
	x_T(t_T) = \frac{\sqrt{1+(a_0t_T)^2}-1}{a_0} 
\end{equation}
Se si prende il limite di basse velocit\`a si avr\`a \( a_0t_T << 1 \), da cui si ottiene
\[ x_t(t_T) \sim \frac{1}{2}a_0t_T^2 \]
che corrisponde alla formula newtoniana.

\subsubsection{Achille e la lepre}
Uguagliando la posizione di una particella accelerata da ferma a partire da $x_0$ alla posizione di un raggio luminoso partito dall'origine, si ottiene
\[ t_T = \frac{\sqrt{1+(a_0t_T)^2}-1}{a_0} + x_0 \] 
\[ a_0t - x_0a_0 +1 = \sqrt{1+(a_0t_T)^2}-1 \]
con condizione \( a_0t - x_0a_0 +1 >0 \) \todo. Procedendo,
\[ t = \frac{x_0(x_0a_0 -2)}{2(1-a_0x_0)} \]
che ha soluzione per \( 1 < x_0a_0 <2 \).



\subsubsection{Tempo proprio}
Con \(\de s^2 = -c^2\dt^2 + \dx^2 \):
\[ ic\de\tau = \de s = ic\dt_T \sqrt{1-v^2(t_T)} \]
\[ \tau = \int_0^{t_T} \sqrt{1-v^2(t)} \dt = \int \frac{1}{\sqrt{1+(a_0t)^2}} \dt \]
Sostituendo \( a_0t = \cosh z \)
\[ \tau = \int \frac {\de z}{a_0} = \frac{\sinh^{-1}(a_0t_T)}{a_0} = \frac{\ln({\sqrt{1+(a_0t_T)^2}+a_0t_T)}}{a_0}\]

\subsubsection {$10^9$ anni-luce} 
%Per percorrere una distanza di $10^9 \mathrm{ly}$ con accelerazione da fermo di \(g=9.8m/s^2=1.030ly/y^2\), usando la formula \ref{leggeoraria}, occorrono
%\[ \sqrt{\frac{d}{c}^2 + 2\frac{d}{g}} \simeq 2.998\cdot10^18y \]
%cui corrisponde un tempo proprio \( \tau \simeq 41.972 y \).
Utilizzando la formula \ref{leggeoraria}
\[ 10^9  = \frac{ \sqrt{1 +(1.030t)^2} -1 }{1.030} \]
da cui \( t_T\sim 10^9 \).
Utilizzando la formula del tempo proprio,
\[ \tau = \frac{ \ln(\sqrt{1+(a_0t_T)^2} + a_0t_T )}{a_0} \sim 20.82 y \]

\subsubsection {Perch\`e non andiamo su Giove?}
Con le formule della meccanica classica,
\[ t_{TOT} = 4\cdot \sqrt{\frac{2x_{TM}}{g}} = 2.8558587119250753 y \]
In relativit\`a ristretta, dove 'lh' sono le ore-luce,
\[ t_{TM} = 260.627804384274lh \]
\[ v_{TM} = 0.9999999999999941 c \]
Modificando opportunamente la formula \ref{veloc} per velocit\`a iniziale non nulla, si ottiene
\[ v(t_T) = \frac{a_0 t_T + \tan\arcsin(v_0)          }
	{\sqrt{1+( a_0t_T + \tan\arcsin(v_0)           )^2 }}  \]
\[ v(t_T) = \frac{a_0 t_T + \frac{v_0}{\sqrt{1-v_0^2}} }
	{\sqrt{1+( a_0t_T + \frac{v_0}{\sqrt{1-v_0^2}}  )^2 }}  \]
E per la legge oraria
\[ x_T(t_T) = \frac{\sqrt{1+ (-gt_T + \frac{v_0}{\sqrt{1-v_0^2}})^2 } - 
	\sqrt{1+  (\frac{v_0}{\sqrt{1-v_0^2}})^2}     }{-g} \]
Invertendo:
\[ t = \frac{ \frac{v_0}{\sqrt{1-v_0^2}} + \sqrt{ (\frac{v_0}{\sqrt{1-v_0^2}})^2 - 2gx \sqrt{1+( \frac{v_0}{\sqrt{1-v_0^2}}    )^2     }  } }
             {  g   }\]
\todo risultati bruttissimi

2.99792458e+18  anni
tempo proprio:
41.9715540832882 anni




\subsubsection{Razzo relativistico}
Considero il sistema $\mathcal{I}$ in cui il razzo \`e fermo e i sistemi $\mathcal{E}$, in cui \`e ferma la $\de m$ espulsa, e $\mathcal{J}$, in cui \`e fermo il razzo propulso con massa $m-\de m$. Nel sistema $\mathcal{I}$:
\[ \delta m v_e \gamma(v_e) = (m - \de m) \de v \gamma(\de v) \sim m \de v \]
\[ \frac{\delta m}{m} = \frac{1}{v_e \gamma(v_E)} \frac{\de v}{\de\tau} \de\tau \]
Considerando che \( \delta m \gamma(v_e)  = -\de m \) e che \( \frac{\de v}{\de t} = a_0 \),
\[ \frac{\de m}{m} = -\frac{\de v}{\de t} \frac{\de t}{\de \tau} \frac{\tau}{v_e} = -\frac{a_0}{v_e\sqrt{1-v_e^2}}\de \tau \]
da cui infine
\[ m(\tau) = e^{-\frac{a_0\tau}{v_e \sqrt{1-v_e^2}}} \]
%\[ m = m_0 e^{\frac{\frac{\de v}{\de\tau}}{v_e \gamma(v_e) }} \]
%\[ m = m_0 e^{\frac{a_0\tau}{v_e \gamma(v_e) }} \]

\subsubsection{Campo elettrico}





\subsection{ Campo elettrico di una particella carica senza massa}
\subsubsection{Quadripotenziale}
Considerando la gauge in cui le componenti spaziali di $A^\mu$ sono nulle (in quanto il campo magnetico prodotto da una particella ferma \`e nullo), $A^0$ puo' essere ricavato, a meno di costanti, considerando che la sua derivata \`e il campo elettrico, che sara\`a un vettore (D-1)-dimensionale. Per il teorema di Gauss,
\[ \int_\Sigma \vec{E}\cdot\vec{\Sigma} = \frac{\rho}{\epsilon_0} =  (cost) \cdot \frac{r^{D-1}}{r^k}\Omega \]
E siccome la dipendenza da r deve cancellarsi \( E_i = \frac{\Gamma(\frac{D-1}{2})}{2\pi^{\frac{D-1}{2}}\epsilon_0} \frac{1}{r^{D-2}} \), quindi
\[ A = (-\frac{\Gamma(\frac{D-1}{2})}{(D-1)2\pi^{\frac{D-1}{2}}\epsilon_0} \frac{1}{r^{D-1}},\vec{0}) \]
\subsubsection{Tensore elettromagnetico}
Le componenti non nulle del tensore $F_{\mu\nu}$ sono le $F_{i0}=-F_{0i}=E_i$.

\subsubsection{Coordinate cono-luce}
Il cambio di coordinate \`e
\[ \Upsilon_\xi^\mu = 
\begin{pmatrix}
	1/\sqrt{2} & 1/\sqrt{2}  \\
	1/\sqrt{2} & -1/\sqrt{2} \\
\end{pmatrix} 
\]
e il tensore elettromagnetico trasforma nel modo seguente:
\[ \tilde{F}_{\xi\varsigma} = \Upsilon_\xi^\mu \Upsilon_\varsigma^\nu F_{\mu\nu} = 
\begin{pmatrix}
	0    & E_1 \\
	-E_1 & 0   \\
\end{pmatrix}
\]

Se cambio la velocit\`a:
\[ x^{\pm} = \gamma(1-\beta) x^{\pm} = \sqrt{\frac{1-\beta}{1+\beta}} \]








































