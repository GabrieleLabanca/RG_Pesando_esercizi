\subsection{Osservatori in Minkowsky}
Data la metrica 
\[ \de s^2 = -\de t^2 + \dx^2 \]
il cambio di variabili
\[ x = x_0 + vt_0 ; \;\;\; t = t_0 \]
porta, in forma canonica, a 
\[ \de s ^2 = (v^2 -1)\left( \de t_0 + \frac{v}{v^2 -1} \dx_0\right)^2   - \frac{1}{v^2-1} \dx_0^2 \]
da cui si identifica facilmente
\[ \de l^2 = \frac{1}{v^2-1} \dx_0^2 \]
Si pu\`o quindi definire una nuova coppia di variabili, separando la parte spaziale e quella temporale: \( \de X = \de l \).
\[ \de s^2 = -\de T^2 + \de X^2 \;\;\; \Rightarrow \;\;\; X = \frac{x_0}{\sqrt{v^2-1}} \;\;\;;\;\;\; T =\sqrt{v^2 -1}\left( t_0 + \frac{v}{v^2 -1} x_0\right) \]



\subsection{Metrica di Rindler}
Presa la metrica 
\[ \de s^2 = -\kappa^2x^2\de t^2 +\de x^2 \]
si calcolano le geodetiche
\[ 0 = \ddot{x} + k^2x\dot{t}^2 = \ddot{x} + \Gamma^x_{tt} \dot{t}^2 \;\;\; ;\;\;\; 0= \ddot{t} + \frac{2}{x} \dot{x}\dot{t} = \ddot{t} + \Gamma^t_{xt}\dot{x}\dot{t} \]
da cui 
\[ \Gamma^x_{tt} = k^2x \;\;\;;\;\;\;  \Gamma^t_{xt} = \frac{1}{x} \]
Per le traiettorie a \(x=\bar{x}=\mathrm{cost} \)
\[ \gamma(t) = (t,\bar{x}) \]
si ha 
\[ \dot{\gamma} = \partial_t \gamma= e_t \Rightarrow \|\dot{\gamma}\|^2 =-\kappa^2x^2  \]
da cui 
\[ u_\mu = \frac{\dot{\gamma}}{\sqrt{-\|\dot{\gamma}}} = \frac{e_t}{k\bar{x}} \]
L'accelerazione \`e quindi
\[ a = \nabla_ u = \frac{1}{k^2\bar{x}^2} \nabla_t e_t =  \frac{1}{k^2\bar{x}^2} \Gamma_{tt}^x e_x = \frac{e_x}{\bar{x}} \]
Nel caso di osservatori che si muovano con moto accelerato, la metrica non \`e pi\`u stazionaria, quindi gli osservatori non sono sincronizzabili e la metrica non \`e riconducibile alla forma canonica.






\subsection{Schwartzschild per osservatori in moto}
Data la metrica di Schwartzschild in D dimensioni
\[  \de s^2 = -c^2 \left( 1- \left(\frac{r_s}{r}\right)^{D-3}\right) \de t^2 + \frac{\de r^2}{1- \left(\frac{r_s}{r}\right)^{D-3}} + r^2 (\de \theta_{D-2} + ...) \]
si calcola il limite di campo debole della sua azione: si considera \( g_{\mu\nu} = \eta_{\mu\nu} + h_{\mu\nu} \) dove $\eta$ \`e la metrica piatta e $h$ rappresenta le perturbazioni gravitazionali, tali che siano infinitesime e dello stesso ordine tra loro.
\[ S = -mc \int \de \lambda \sqrt{- c^2  \left( 1- \left(\frac{r_s}{r}\right)^{D-3} \right)  \dot{t}^2 + g_{ij} \dot{x}^i \dot{x}^j} \]
dove \( \left(\frac{r_s}{r}\right)^{D-3} = \mathcal{O}(|h|) \).
Prendendo \( \lambda = t \) e moltiplicando per 
\( (1-v^2/c^2)^{1/2} (1-v^2/c^2)^{-1/2} \sim \left(1-\frac{v^2}{2c^2}\right) \left(1+\frac{v^2}{2c^2}\right)\)
(dove il quadrato di un vettore indica la sua norma al quadrato) si arriva a 
\[ S = -mc^2 \int \de t  \left(1-\frac{v^2}{2c^2}\right) \left( 1 -\frac{1}{2} \left(\frac{r_s}{r}\right)^{D-3} + \mathcal{O}(\frac{v^2}{2c^2}\cdot |h|) \right) \]
\[ = -mc^2 \int \de t  \left(1-\frac{v^2}{2c^2} -\frac{1}{2} \left(\frac{r_s}{r}\right)^{D-3} \right)\]
\[ = \int \de t \left(  -mc^2  + \frac{1}{2}mv^2 + \frac{1}{2} mc^2\left(\frac{r_s}{r}\right)^{D-3} \right)\]
Identificando i primi due addendi come il termine di massa e quello cinetico, si pu\`o identificare il restante come potenziale gravitazionale, da cui
\[ \frac{1}{2} mc^2\left(\frac{r_s}{r}\right)^{D-3} = -m\phi_{grav} \;\;\Rightarrow \;\;r_s^{D-3} = \frac{2MG_D}{c^2} \]

Si pu\`o definire il cambiamento di coordinate
\[ r^2 = x^2 + \vec{x}^2_\perp \;\;\; ; \;\;\; \cos\theta_{D-2} = \frac{x_{D-1}}{r}  \;\; ; \;\; \cos\theta_{D-3} = \frac{x_{D-1}}{r\cos\theta_{D-2}} \;\; ...\] 
e si vede che 
\[ \sum \de x_i^2 = \de x^2 + \de x^2_\perp = \de r^2 + r^2(\de \theta_{D-2} + \de \theta_{D-3} \sin^2\theta_{D-2} + ... )\]
da cui, raccogliendo $\de r$,
\[  \de s^2 = -c^2 \left( 1- \left(\frac{r_s}{r}\right)^{D-3} \right) \de t^2 
    + \frac{     \left(\frac{r_s}{r}\right)^{D-3} }
           { 1-  \left(\frac{r_s}{r}\right)^{D-3} } \de r^2 
    +  \de x^2 + \de x^2_\perp  \]


\subsubsection{Osservatori in moto}
L'analogo di una trasformazione di Lorentz si ha con
\[ ct = \gamma \frac{1+\beta}{2}U + \gamma \frac{1-\beta}{2}V \;\;\; ; \;\;\; x = \gamma \frac{1+\beta}{2}U - \gamma \frac{1-\beta}{2}V \]
dove U e V sono le coordinate di cono luce. Siccome si vuole determinare la metrica per un osservatore in moto alla velocit\`a della luce,  si pone il limite \( \beta \rightarrow 1 \).
Si scrive la trasformazione dei termini in $\de t$ e $\dx$, per poi imporre le condizioni desiderate:
\[ -c^2 \left( 1- \left(\frac{r_s}{r}\right)^{D-3} \right) \de t^2 +  \de x^2 \] 
\[ \rightarrow \;\;\; \gamma^2 \frac{(1+\beta)^2}{4} \de U^2 + \gamma^2 \frac{(1-\beta)^2}{4} \de V^2 - \de U\de V + \left(\frac{r_s}{r}\right)^{D-3}(\gamma^2 \frac{(1+\beta)^2}{4} \de U^2 + \gamma^2 \frac{(1-\beta)^2}{4} \de V^2 - \de U\de V) \]
Si nota subito che tutti i coefficienti di $\de V$ tendono a zero. Inoltre per \(U\neq 0\)
\[ r = \sqrt{ \gamma^2 (\frac{(1+\beta)^2}{4} U^2 +  \gamma^2 (\frac{(1+\beta)^2}{4} V^2 - UV +   x^2_\perp} \]
che nel limite tende a infinito a causa del $\gamma$ davanti a $U^2$, per cui tutti i termini $\left(\frac{r_s}{r}\right)^{D-3}$ si annullano e rimane
\[ \de s^2 = -\de U\de V + \de x^2_\perp \]
Nel caso in cui \(U=0\), invece, 
\[ r = \sqrt{  x^2_\perp } = x_\perp\]
e 
\[ \de s^2 = \gamma^2  \left(\frac{r_s}{x_\perp}\right)^{D-3} \de U^2 + \mathrm{o}(\de U^2) \]
che diverge nel limite.

\subsubsection{Metrica di Aichelburg-Sexl}
I termini in $\de t$ e $\dx$ si possono riscrivere all'ordine di $\de U^2$, tenendo conto che il termine in $V^2$ svanisce, come
\[ -\de U\de V + r_s^{D-3} \left[ \frac{\gamma^2}{ (\gamma^2 U^2 - UV + x_\perp^2)^{(D-3)/2} } \de U^2 \right] \]
e usando il limite noto \ref{eq:limND}
\[ \left[ \frac{\gamma^2}{ (\gamma^2 U^2 - UV + x_\perp^2)^{(D-3)/2} } \right] =  N_D \frac{\delta{U}}{-UV + x_\perp^{D-4}} =  N_D \frac{\delta{U}}{ x_\perp^{D-4}} \]
inoltre si impone 
\[ \mathrm{lim} M \gamma = M_* \]
che ha il significato di una massa limite, non infinita, in modo che abbia significato fisico;
per cui 
\[r_s \rightarrow r_{s*} \]
Unendo i risultati 
\[ \de s^2 = -\de U\de V + N_D r_{s*}^{D-3} N_D \frac{\delta{U}}{ x_\perp^{D-4}} \de U^2 + \de x_\perp^2 \]
cio\`e tale metrica \`e analoga a quella piatta, fatta eccezione per il supporto della $\delta$, cio\`e la bisettrice del II e IV quadrante nel piano (X,cT).

